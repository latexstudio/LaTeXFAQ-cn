% Copyright (C) 2018 by latexstudio <http://www.latexstudio.net>
%
% This program is free software: you can redistribute it and/or modify
% it under the terms of the GNU General Public License as published by
% the Free Software Foundation, either version 3 of the License, or
% (at your option) any later version.
%
% This program is distributed in the hope that it will be useful,
% but WITHOUT ANY WARRANTY; without even the implied warranty of
% MERCHANTABILITY or FITNESS FOR A PARTICULAR PURPOSE.  See the
% GNU General Public License for more details.
%
% You should have received a copy of the GNU General Public License
% along with this program.  If not, see <http://www.gnu.org/licenses/>.
%

\section{参考文献篇}


\faq{参考文献中的特殊字符或字母}{}


\faq{BibTeX 不理解的作者列表}{}

BibTeX 只支持三种姓名格式: * First von Last * von Last, First * von
Last, Jr, First

多个姓名之间必须使用``and''连接,如

\begin{verbatim}
author = {Knuth, Donald E. and Lamport, Leslie},
\end{verbatim}


\faq{BibTeX 排序和名字前缀}{}


\faq{BibTeX 中的大写字母}{}

英文标题中常使用的大小写方式有:

\begin{enumerate}
\def\labelenumi{\arabic{enumi}.}

\item
  Title case:
  句首字母大写,并且除冠词、连词和短介词以外的词首字母大写,这里说的``短''介词一般指不超过
  4 个字母的介词。比如``The Quick Brown Fox Jumps over the Lazy Dog'';
\item
  Sentence case:
  句首字母和一些专有名词的首字母大写,同普通的英文句子大小写方式一样,如``The
  quick brown fox jumps over the lazy dog''。
\end{enumerate}

BibTeX 根据 bst 样式文件可以将题名保留原大小写,或转为 sentence
case。所以用户在 bib 数据库中著录标题的正确方式是,统一使用 title
case,并将需要专有名词用大括号括起来。

\begin{verbatim}
title = {Finite Element Methods for {Maxwell's} Equations},
\end{verbatim}

注意尽量避免将一个词中个别字母用大括号括起来,如``\{M\}axwell's'',这可能会导致字母的间距有问题,建议将整个词括起来,如``\{Maxwell's\}''。


\faq{如何选择参考文献的风格}{}

参考文献的风格一般是期刊或会议模板指定 bst
的,作者应仔细阅读投稿要求和模板使用说明,根据规定使用合适的
bst。通常有以下方式:

\begin{enumerate}
\def\labelenumi{\arabic{enumi}.}

\item
  在文档中声明 |\bibliographystyle{ieeetran}|
\item
  在模板的文档类选项中使用合适的参数,如 |\documentclass[authoryear]{ustcthesis}|。
\end{enumerate}


\faq{BibTeX 参考文献数据库}{}

BibTeX 的 bib 文件是一个记录已阅文献的数据库,但是通常不建议手动编译 bib
文件,建议:

\begin{enumerate}
\def\labelenumi{\arabic{enumi}.}

\item
  使用 JabRef 或 Zotero 等文献管理工具导出 bib 文件创
\item
  使用 \href{https://scholar.google.com/}{Google Scholar} 或
  \href{https://cn.bing.com/academic}{Bing 学术}导出 bib 条目建
\end{enumerate}


\faq{创建参考文献风格}{}

BibTeX 的风格文件 bst
是使用一种后缀语言写的代码,如果对编程能力比较自信的话,可以阅读 BibTeX
的文档 btxdoc 和 btxhak,btxbst.doc 文件提供了标准 bst
风格的代码注释,另外还可以阅读 ttb 和 The LaTeX Companion 等资料。

如果不习惯 bst 的编程语言,可以使用 custom-bib 工具,在命令行下运行latex
makebst,回答一系列问题生成自己的bst。

另外还可以考虑使用 biblatex,它提供更方便的接口用于自定义参考文献格式。


\faq{参考文献中的数字格式}{}

参考文献表中的数字格式是由 @biblabel
控制的,可以通过重定义该命令来修改格式。比如将数字修改为左对齐:

\begin{verbatim}
\makeatletter
\renewcommand\@biblabel[1]{[#1]\hfill}
\makeatother
\end{verbatim}


\faq{BibTeX文献条目列表}{}

科技论文通常要求参考文献表中的文献必须在正文中引用,但是在某些特殊情况下仅需要列出
bib 数据库中的文献,可以使用 |\nocite{*}|
命令列出调用的bib中所有条目,或者使用类似 |\nocite{ref1,ref2,ref3}| 命令列出需要显示的条目。


\faq{制作参考文献的HTML}{}


\faq{BibTeX中的多字母缩写}{}


\faq{多个参考文献表}{}

natbib宏包与Donald Arseneau和Niel
Kempson编写的chapterbib宏包兼容,该宏包允许在一个文档内有多个独立的参考文献列表。通常用法是一本书的各章有独立的参考文献列表,尤其是在各章由不同作者独立编写时。


\faq{同一位置多文献引用}{}

只需要将多篇文献的bibkey用英文半角逗号分隔写在一个cite指令的选项里即可。如:

\begin{verbatim}
\cite{knuth84,lamport86}
\end{verbatim}


\faq{非英文参考文献条目}{}

什么叫非英文参考文献条目?是指bibkey么?一般不建议用中文,处理好编码格式,无殊。
中文的参考文献条目,与英文条目并没有什么差别,只是注意编码。目前处理中文推荐用xelatex
编译 utf8 编码的文件。因此中文的 bib 条目也应该用 utf8 编码。


\faq{BibTeX 文献手写很困难,有没有什么工具能够生成?}{}

多数时候,我们无需自己手写 BibTeX 文献条目。从
\url{https://scholar.google.com/}、\url{https://academic.microsoft.com/}、
\url{https://cn.bing.com/academic?mkt=zh-CN}
或者期刊、数据库的网站上都能够导出 BibTeX 文献条目。 老牌的文献管理软件
EndNote 也支持生成 BibTeX 格式的数据库,详情见
官网\url{https://endnote.com/}。 开源软件 JabRef 甚至支持 BibTeX
文献条目的导入、导出和管理,详情见 官网\url{http://www.jabref.org/}。
Zetero 使用起来也非常方便,详情见官网 \url{https://www.zotero.org/}。
谷歌学术、知网、百度学术、万方数据库等在线数据库也是可以支持导出 .bib
文件的,至于哪家的数据条目更全,就得你自己去甄别了。


\faq{如何使用 BibTeX 排版参考文献}{}

\begin{itemize}

\item
  准备一份 BibTeX 数据库,假设数据库文件名为 books.bib,和 LaTeX
  源代码一般位于同一个目录下。
\item
  在源代码中添加必要的命令,如 |\bliographystyle{abbrv}|,|\bibliography{books}|。假设源代码名为
   demo.tex。其中,\cs{bibliographystyle}设定参考文献的格式。
   \cs{bibliography}
  告诉系统使用哪个数据库和参考文献列在哪个位置。
\item
  写好了以上两个文件之后,我们就可以开始编译了。例如在命令行中执行以下命令
\end{itemize}

\begin{verbatim}
xelatex demo
bibtex demo
xelatex demo
xelatex demo
\end{verbatim}

或者选择一个可以自动检测是否有参考文献的编辑器,如果有,它会自动执行以上四个命令,但是有时候会遇到检测不到的情况,这时你只需要清理一下辅助文件即可。


\faq{如何将参考文献条目录入到正文中}{}

理工科类论文很少用。


\faq{bib文件的重建}{}

用文本编辑器如Notepad++, Sublime
Text或WinEdt或专门文献管理软件JabRef,BibDesk等创建文件,改名为 ref.bib
文件,往里头添加参考文献目录。参考如下:
% \includegraphics{https://images-cdn.shimo.im/VKQ8uAycksg1zPlo/image.png!thumbnail}
在.bib文件中,可以采用 TeXStudio 提供的参考文献格式,在自行修改内容
% \includegraphics{https://images-cdn.shimo.im/0OgCsRQoufMTDJ75/1.png!thumbnail}
上面的类型有两种选择 BibTeX 和 BibLaTeX ,后者的选择更为广泛。
参考文献一般不自己书写,而是有可以直接导入。 一般直接 Google
学术搜索出来的文献或者引用知网,如下:
% \includegraphics{https://images-cdn.shimo.im/L1fAEZmW9tYDVTYT/VRI1FEC62J_C6_QSK_P0_0.png!thumbnail}
点击上图红圈的引号-\textgreater{}
% \includegraphics{https://images-cdn.shimo.im/N8tFzuXsCM8rOPjF/image.png!thumbnail}
在点击最左侧的 BibTeX -\textgreater{}
% \includegraphics{https://images-cdn.shimo.im/81Z6BGei8ycQf1uK/image.png!thumbnail}
将其复制黏贴到你的 ref.bib 文件中即可。
在知网上的文献查询需要下载安装如下软件:
% \includegraphics{https://images-cdn.shimo.im/ZsikCVGdjGIKBqSN/image.png!thumbnail}
两个都装好了之后,该软件需要自行注册登陆使用。
然后打开知网,会看到如下:
% \includegraphics{https://images-cdn.shimo.im/DVEoaSyHJKwmbSjH/2.png!thumbnail}
右上角红圈圈到的就是为浏览器安装的 Zotero Connector插件,在此需要打开
Zotero
软件,点击之后显示下图,选择需要的文献。
% \includegraphics{https://images-cdn.shimo.im/w4eu1WOehS05gJ0g/image.png!thumbnail}
然后 Zotero 软件如下显示
% \includegraphics{https://images-cdn.shimo.im/VFUjYs5MvKQz522e/image.png!thumbnail}
然后文件-\textgreater{}导出文献库-\textgreater{}导出格式 BibTeX
确定保存生成的bib文件,可以将这个 bib 文件中的参考文献全部复制黏贴到你的
ref.bib
文件中,也可以单独作为一个新的bib文件,在正文区则需要添加多个bib文件就可以,用命令
\begin{verbatim}
\bibliography{test,ref}
\end{verbatim}

,多个bib文件用逗号分隔即可。同时为引用的参考文献需要命令 \nocite{*}
来将未引用的文件全部排版出来。 注:百度学术、万方数据库等也支持导出 .bib
文件。


\faq{如何减少参考文献条目行间距}{}

文献条目间距为 \cs{itemsep},默认值4.5pt plus 2pt minus
1pt,可通过指令 |\addtolength{\itemsep}{距离}| 调整。


\faq{按照章节分开参考文献条目}{}

可看看chapterbib宏包。 

\#\# 引文的排序及压缩

这个取决于使用的宏包,常用的natbib宏包可以使用sort或者sort\&compress选项激活相应的排序或排序并压缩功能。
\#\# 引文列表排序

这个取决于bst,一般模板都有指定的bst。 \#\# BibTeX中过长的字符串 \#\#
按照``unsrt''规则的目录重排序 \#\# BibTeX参考文献中的URL

调用url或者xurl宏包即可正常使用url,也可以看看href宏包。 \#\# 基于Plain
TeX的BibTeX的使用 \#\# 常用的biblatex参考文献样式

biblatex除了可以应用自带的标准样式外,还可以使用其他作者提供的第三方样式,这里介绍一些常用的样式:
* 国外常用 * APA * MLA * 国内 * GB7714-2015
样式名\textbar{}用法\textbar{}对应的bibtex样式\textbar{}作者介绍\textbar{}样式说明\textbar{}
:----:\textbar{}:----:\textbar{}:----:\textbar{}:----:\textbar{}:----:\textbar{}
trad-plain\textbar{}\texttt{\textbackslash{}usepackage{[}style=trad-plain{]}\{biblatex\}}\textbar{}plain\textbar{}MarcoDaniel
and
MoritzWemheuer,后者是biblatex维护者之一\textbar{}将引文按字母顺序排序,比较次序为作者姓氏、出版年份和题名,如果不能顺序,将以在正文中的引用顺序为准。\textbar{}
trad-unsrt\textbar{}\texttt{\textbackslash{}usepackage{[}style=trad-unsrt{]}\{biblatex\}}\textbar{}unsrt\textbar{}MarcoDaniel
and
MoritzWemheuer\textbar{}按照在正文中引用文献的先后顺序排列文献,其排版格式与trad-plain基本相同\textbar{}
trad-alpha\textbar{}\texttt{\textbackslash{}usepackage{[}style=trad-alpha{]}\{biblatex\}}\textbar{}alpha\textbar{}MarcoDaniel
and
MoritzWemheuer\textbar{}用文献的作者姓氏前三个字母加出版年份的后两位数作为文献序号,如果出现相同的序号,则会根据排序结果在序号后追加字母以示区别,排序方法和排版格式与trad-plain相同\textbar{}
trad-abbrv\textbar{}\texttt{\textbackslash{}usepackage{[}style=trad-abbrv{]}\{biblatex\}}\textbar{}abbrv\textbar{}MarcoDaniel
and MoritzWemheuer\textbar{}将文献中作者名和月份名的拼写改为缩写,
显得文献信息紧凑简洁, 其排序方法和排版格式与trad-plain相同\textbar{}
ieee\textbar{}\texttt{\textbackslash{}usepackage{[}style=ieee{]}\{biblatex\}}\textbar{}IEEEtran\textbar{}Joseph
Wright,biblatex
维护者之一\textbar{}国际电气电子工程师协会IEEE期刊文献格式\textbar{}
apa\textbar{}\texttt{\textbackslash{}usepackage{[}style=apa{]}\{biblatex\}}\textbar{}apalike\textbar{}Philip
Kime,biblatex 作者之一\textbar{}American Psychological Association
的文献格式\textbar{}
Chicago\textbar{}\texttt{\textbackslash{}usepackage\{biblatex-chicago\}}\textbar{}Chicago\textbar{}David
Fussner\textbar{}for the Chicago Manual of Style\textbar{}
iso-numeric\textbar{}\texttt{\textbackslash{}usepackage{[}style=iso-numeric{]}\{biblatex\}}\textbar{}
\textbar{}Michal Hoftich\textbar{}ISO690 international standard numeric
system\textbar{}
iso-iso-authoryear\textbar{}\texttt{\textbackslash{}usepackage{[}style=iso-iso-authoryear{]}\{biblatex\}}\textbar{}
\textbar{}Michal Hoftich\textbar{}ISO690 international standard
nameanddate system,so-called Harvard style\textbar{}
gb7714-2015\textbar{}\texttt{\textbackslash{}usepackage{[}style=gb7714-2015{]}\{biblatex\}}\textbar{}gbt7714-unsrt.bst
by zepinglee\textbar{}hushidong\textbar{}中文文献著录标准 GB/T 7714-2015
顺序编码制\textbar{}
gb7714-2015ay\textbar{}\texttt{\textbackslash{}usepackage{[}style=gb7714-2015ay{]}\{biblatex\}}\textbar{}gbt7714-plain.bst
by zepinglee\textbar{}hushidong\textbar{}中文文献著录标准 GB/T 7714-2015
著者年份制\textbar{}
caspervector\textbar{}\texttt{\textbackslash{}usepackage{[}style=caspervector{]}\{biblatex\}}\textbar{}
\textbar{}Casper vector\textbar{}一种中文文献格式\textbar{}
nature\textbar{}\texttt{\textbackslash{}usepackage{[}style=nature{]}\{biblatex\}}\textbar{}
\textbar{}Joseph Wright\textbar{}for Nature\textbar{}
science\textbar{}\texttt{\textbackslash{}usepackage{[}style=science{]}\{biblatex\}}\textbar{}
\textbar{}Joseph Wright\textbar{}for Science\textbar{}
chem-acs\textbar{}\texttt{\textbackslash{}usepackage{[}style=chem-acs{]}\{biblatex\}}\textbar{}
\textbar{}Joseph Wright\textbar{}covers most American Chemistry Society
journals\textbar{}
chem-angew\textbar{}\texttt{\textbackslash{}usepackage{[}style=chem-angew{]}\{biblatex\}}\textbar{}
\textbar{}Joseph Wright\textbar{}covers Angewandte Chemie Chemistry--A
European Journal.\textbar{}
chem-biochem\textbar{}\texttt{\textbackslash{}usepackage{[}style=chem-biochem{]}\{biblatex\}}\textbar{}
\textbar{}Joseph Wright\textbar{}covers Biochemistry and asmallnumber of
other American Chemistry Society journals\textbar{}
chem-rsc\textbar{}\texttt{\textbackslash{}usepackage{[}style=chem-rsc\ {]}\{biblatex\}}\textbar{}
\textbar{}Joseph Wright\textbar{}covers all Royal Society of Chemistry
journals\textbar{}
phys\textbar{}\texttt{\textbackslash{}usepackage{[}style=phys{]}\{biblatex\}}\textbar{}
\textbar{}Joseph Wright\textbar{}for AIP and APS\textbar{}
nejm\textbar{}\texttt{\textbackslash{}usepackage{[}style=nejm{]}\{biblatex\}}\textbar{}
\textbar{}MarcoDaniel\textbar{}for New England Journal of
Medicine\textbar{}
mla\textbar{}\texttt{\textbackslash{}usepackage{[}style=mla{]}\{biblatex\}}\textbar{}
\textbar{}James Clawson\textbar{}for Modern Language
Association\textbar{}
authortitle-dw\textbar{}\texttt{\textbackslash{}usepackage{[}style=authortitle-dw{]}\{biblatex\}}\textbar{}
\textbar{}Dominik Waßenhoven\textbar{}for Humanities\textbar{}
footnote-dw\textbar{}\texttt{\textbackslash{}usepackage{[}style=footnote-dw{]}\{biblatex\}}\textbar{}
\textbar{}Dominik Waßenhoven\textbar{}for Humanities\textbar{}


\faq{使用超链接,如何去除颜色边框?}{}

直接在引用 hyperref 宏包的时候使用以下命令之一

\begin{verbatim}
\usepackage[hidelinks]{hyperref}
\usepackage[colorlinks]{hyperref}
\end{verbatim}

第一种方法是隐藏链接,即隐藏颜色和边框。
第二种方法是用不同颜色来替换默认的边框强调超链接的方式,但是这种方法会使得链接具有不同的颜色。如果需要设置各种链接的颜色可以参考
hyprref
的说明文档,值得庆幸的是,该宏包已经有了一个\href{https://github.com/latexstudio/LaTeXPackages-CN/blob/master/hyperref/hyperref-zh-cn.pdf}{中文翻译版}。


\faq{参考文献列表行距如何设置?}{}

设置好文献条目间距 \cs{itemsep} 即可。


\faq{参考文献编号如何左对齐,右对齐?}{}


\faq{插入参考文献列表有几种方式?如何定义其样式?如何定义正文引用样式?}{}


\faq{不同 journal 给出的 bibtex 文件格式不一致,如何批量快速格式化多个 .bib 文件}{}
