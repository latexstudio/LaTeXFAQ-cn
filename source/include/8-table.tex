\section{表格篇}
%
%
%\begin{faq}{如何指定表格的总宽度}
%
%可以看看tabularx、tabu等宏包。
%\end{faq}
%
%
%\begin{faq}{指定列宽度的表格如何使单元格内容居中}
%
%指定宽度的表格列一般采用 p\{\}
%形式的列格式,这种列格式下,表格内容是两端对齐的,如果想使其成为居中对齐需要借助
%array 宏包提供的功能,示例如下:
%
%\begin{verbatim}
%\begin{tabular}{c|>{\centering\arraybackslash}p{4cm}}
%\hline
%1  &  3.530  \\
%2  &  456.0  \\
%3  &  78.945 \\
%4  &  3.65   \\
%\hline
%\end{tabular}
%\end{verbatim}
%
%而 \verb|>{}p{}|
%这样的格式在文档的应用过程中是非常不方便的,array 宏包同时提供了
%\cs{newcolumntype} 宏命令可以将其定义为一个较为简短的格式,如:
%
%\begin{verbatim}
%\newcolumntype{z}[1]{>{\centering\arraybackslash}p{#1}}
%\end{verbatim}
%
%从而可以在正文中使用
%
%\begin{verbatim}
%\begin{tabular}{c|z{4cm}}
%\hline
%
%1  &  3.530  \\
%2  &  456.0  \\
%3  &  78.945 \\
%4  &  3.65   \\
%\hline
%\end{tabular}
%\end{verbatim}
%
%类似的,采用 \cs{raggedright} 或
%\cs{raggedleft} 替换\cs{centering} 可以使得单元格内容变成左对齐或右对齐。
%\end{faq}
%
%
%\begin{faq}{tabularx 中的 X
%  列格式如何居中对齐}
%
%同样采用 array 宏包的 \verb|>|\marg{format} 方法,并利用
%\cs{newcolumntype} 定义新的列格式,如:
%
%\begin{verbatim}
%\usepackage{array,tabularx}  % this line in preamble
%\newcolumntype{Z}{>{\centering\arraybackslash}X} % this line in preamble
%\begin{tabularx}{\linewidth}{ZZ}
%\hline
%
%1  &  3.530  \\
%2  &  456.0  \\
%3  &  78.945 \\
%4  &  3.65   \\
%\hline
%\end{tabularx}
%\end{verbatim}
%\end{faq}
%
%
%\begin{faq}{tabularx 中的 X
%  列格式,当单元格内容发生换行时,如何使同一行其他列的单元格垂直居中对齐?}
%
%对于指定宽度的表格列格式
%p\{\},单元格内一旦进行换行,该单元格同一行内其他列的单元格内容均为垂直方向上顶端对齐,我们可以使用
%array 宏包,以 m\{\} 列格式或者 b\{\} 列格式 替代 p\{\}
%格式即可实现垂直居中对齐或垂直底部对齐。对于 tabularx 中的 X
%列格式,也是采用同样的思路实现,只是这里需要对
%\cs{tabularxcolumn} 宏进行重定义如下:
%
%\begin{verbatim}
%\usepackage{array,tabularx}   % this line in preamble
%\renewcommand{\tabularxcolumn}[1]{m{#1}}  % this line in preamble
%\end{verbatim}
%
%以上则将同行的其他列单元格设置为垂直居中对齐。显然的,垂直底部对齐的设置方法是将重定义宏命令中的
%m\{\#1\} 替换为 b\{\#1\} 即可。
%\end{faq}
%
%
%\begin{faq}{booktabs的三线表,竖线为什么是不连续的?}
%
%宏包的作者为表格线的前后都增加了额外的sep,而且,宏包的作者认为三线表是不应该有竖线的。当然,如果你一定想要使用竖线,不妨以下面两个命令将表格线前后的sep设置为0pt。
%
%\begin{verbatim}
%\usepackage{booktabs} % this line in preamble
%\setlength{\belowrulesep}{0pt}
%\setlength{\aboverulesep}{0pt}
%\end{verbatim}
%\end{faq}
%
%
%\begin{faq}{表格的一列全是公式,有什么办法能输入简单些?}
%
%可以使用 array 宏包,\verb|>{}| 与\verb|<{}|
%可以为一列数据前后加上特定的宏命令。在一列数据前后均加上 \verb|$| 则把这列数据放入数学模式中,举例如下:
%\begin{verbatim}
%\usepackage{array} % this line in preamble
%\begin{tabular}{>{$}c<{$} c}
%\hline
%\multicolumn{1}{c}{function} & value \\
%g(x)                         & 3.65  \\
%f(x)                         & 2.58  \\
%\sin(x)                      & 14.7  \\
%\hline
%\end{tabular}
%\end{verbatim}
%
%第一列数据省去了输入数学模式起止符号 \verb|$| 的痛苦。对于不需要放入数学模式的单元格,比如表头,需要用 
%\verb|\multicolumn{1}{c}{xxx}| 的方式来保护一下,重新指定对齐方式。
%\end{faq}
%
%
%\begin{faq}{我的表格单元格内容是一个列表环境 
%(enumerate/itemize),它和表格横线之间间距好大啊,怎么能把这些间距去掉?}
%
%把列表环境放入到 minipage 环境中即可,即使表格列格式采用的是p{<width>}格式。
%\end{faq}
%
%
%\begin{faq}{如果想让表格中数字小数点对齐要怎么做}
%
%可以借助 @ 的功能,如
%
%\begin{verbatim}
%\begin{tabular}{r@{.}l}
%\hline
%1 & 0 \\
%23 & 1 \\
%\hline
%\end{tabular}
%\end{verbatim}
%
%或者借助 warpcol 宏包提供的功能,如
%\begin{verbatim}
%\documentclass{article}
%\usepackage{warpcol}
%\begin{document}
%\begin{tabular}{P{3.1}P{-2.1}}
%\hline
%\multicolumn{1}{c}{Label 1} & \multicolumn{1}{c}{Label 2} \\
%\hline
%123.4 & -12.3 \\
%12.3 & 12.3 \\
%1.2 & 1.2 \\
%\hline
%\end{tabular}
%\end{document}
%\end{verbatim}
%
%还可以借助 array 和 dcolumn 的配合,如
%
%\begin{verbatim}
%\documentclass{article}
%\usepackage{array,dcolumn}
%\newcolumntype{d}[1]{D{.}{.}{#1}}
%\begin{document}
%\begin{tabular}{cd{3}}
%\hline
%1 & 3.14 \\
%2 & 27.12 \\
%3 & 78.095 \\
%\hline
%\end{tabular}
%\end{document}
%\end{verbatim}
%
%
%还可以借助 array 和 dcolumn 的配合,如
%
%\begin{verbatim}
%\documentclass{article}
%\usepackage{array,dcolumn}
%\newcolumntype{d}[1]{D{.}{.}{#1}}
%\begin{document}
%\begin{tabular}{cd{3}}
%\hline
%1 & 3.14 \\
%2 & 27.12 \\
%3 & 78.095 \\
%\hline
%\end{tabular}
%\end{document}
%\end{verbatim}
%\end{faq}
%
%
%\begin{faq}{表格竖排}
%
%\begin{verbatim}
%\documentclass{ctexart}
%\usepackage[usestackEOL]{stackengine}
%
%\begin{document}
%
%\setlength\normalbaselineskip{11pt}
%\strutlongstacks{T}
%\begin{tabular}{|c|c|c|}
%\hline
%Foo bar & {\Centerstack{ 这 \\ 一 \\ 列 \\ 竖 \\ 排 }} & Foo bar \\
%\hline
%\end{tabular}
%
%\end{document}
%\end{verbatim}
%\end{faq}
%
%
%\begin{faq}{跨页长表格}
%
%\begin{verbatim}
%\usepackage{longtable}
%\end{verbatim}
%
%,做好对长表格跨页时的设置
%\end{faq}
%
%
%\begin{faq}{双栏中表格过大怎么调整?}
%
%\begin{itemize}
%  
%  \item
%  方法一:用 graphicx 宏包提供的 \cs{resizebox} 命令:
%\end{itemize}
%
%\begin{verbatim}
%\resizebox{width}{height}{function}
%\end{verbatim}
%
%resizebox 会放缩 function 中的内容到 width 宽度、height
%高度。需要注意的是,同时指定宽度和高度,一般会导致缩放的内容变形,你也可以指定其中一项,\sout{另一个用!占位,}这样系统会自适应另一个参数,即相当于scale命令。
%* 方法二:用 \texttt{table*} 取代 table 环境,针对的是单栏表格。 *
%方法三:将表格中的字体缩小。 * 方法四:使用横排:使用 rotating 宏包
%\end{faq}
%
%
%\begin{faq}{如何制作列数可变的表格,例如试卷的计分表?}
%
%主要是使用 makecell 和 interfaces-makecell 宏包。下面给出一个 MWE。
%
%\begin{verbatim}
%\documentclass{standalone}
%\usepackage{ctex,calc,makecell,interfaces-makecell,CJKnumb,tabularx,multirow}
%
%\newcounter{TotalPart}
%\newcounter{SubColumn}
%\newcounter{EmptyColumn}
%
%\setcounter{TotalPart}{1}
%
%% 计分表制作
%\newcommand{\ScoreTable}{
%\setcounter{SubColumn}{\value{TotalPart}+2}
%\setcounter{EmptyColumn}{\value{TotalPart}+4}
%\begin{tabularx}{\textwidth}{|*{\theSubColumn}{X<{\centering}|}*{3}{c|}}
%\hline
%\multicolumn{\theSubColumn}{|c|}{\multirow{2}{*}{试卷卷面成绩}}
%& \multicolumn{1}{c|}{\multirow{3}{3em}{课程考核成绩占~\%}}
%& \multicolumn{1}{c|}{\multirow{3}{3em}{平时成绩占\,\%}}
%& \multicolumn{1}{c|}{\multirow{3}{3em}{课程考核成绩}}
%\\
%\multicolumn{\theSubColumn}{|c|}{}
%& & &
%\\
%\cline{1-\theSubColumn}
%\hfill 题 \hfill 号 \hfill~
%& \repeatcell{\theTotalPart}{text=\CJKnumber{\column}}
%& \hfill 小 \hfill 计 \hfill~
%& & &
%\\
%\hline
%\hfill 得 \hfill 分 \hfill~
%& \eline{\theEmptyColumn}
%\\
%\hline
%\end{tabularx}
%}
%
%\begin{document}
%
%\ScoreTable
%
%\end{document}
%\end{verbatim}
%
%CJKnumb 宏包是为了把阿拉伯数字转换为小写汉字序号。 calc
%宏包是为了做四则运算。 tabularx 宏包是为了做列宽自动扩展的表格。
%multirow 宏包是为了合并单元格。 makecell 是制作表格。
%interfaces-makecell
%宏包提供了一系列命令,使得制作可变表格称为可能,同时简化了表格制作。
%\end{faq}
%
%
%\begin{faq}{如何固定表格的总宽度?}
%
%使用 tabular* 环境或 tabularx
%宏包提供的同名环境即可固定表格的总宽度,宏包 tabu
%功能更为强大,用法也更为复杂,可参见相应宏包文档说明。
%\end{faq}
%
%
%\begin{faq}{表格在单元格内如何换行?}
%
%可以通过限制列宽实现,例如下面的例子
%
%\begin{verbatim}
%\begin{tabular}{|c|c|m{50mm}|}%这里用m则必须调用array宏包
%\hline
%a & b & \LaTeX{}表格固定列宽自动换行自动换行自动换行自动换行自动换行\\
%\hline
%a & b & \LaTeX{}表格固定列宽自动换行自动换行自动换行自动换行自动换行\\
%\hline
%a & b & \LaTeX{}表格固定列宽自动换行自动换行自动换行自动换行自动换行\\
%\hline
%\end{tabular}
%\end{verbatim}
%\end{faq}
%
%
%\begin{faq}{如何插入子图/表,各自分别带子标题,不带子标题?}
%
%可参见并列图形、并列子图的排列
%\end{faq}
%
%
%\begin{faq}{如何减小表格,插图,公式,列表等前后空白?}
%
%表格、插图、公式、列表的前后空白很多是由于不良的文本结构引起的,比如太短篇幅的正文,接连几级标题之间没有正文内容,甚至标题之间只有插图和表格等浮动体而没有任何说明的正文,这些都是不好的行文习惯,应杜绝这样的行文方式。此外,一些不良的代码写法也会引入较大的空白,如:
%
%\begin{verbatim}
%\begin{center}
%\begin{figure}
%...
%\end{figure}
%\end{center}
%\end{verbatim}
%
%或者
%
%\begin{verbatim}
%\begin{figure}
%\begin{center}
%\includegraphics{x.pdf}
%\caption{the title}
%\end{center}
%\end{figure}
%\end{verbatim}
%
%而应该采用的方式是:
%
%\begin{verbatim}
%\begin{figure}
%\centering
%\includegraphics{x.pdf}
%\caption{the title}
%\end{figure}
%\end{verbatim}
%
%这是因为 center 环境本身就是一个 list
%列表环境,其与上下文之间就有垂直间距,加上figure
%浮动体与正文之间的间距,插图与正文之间的间距自然就变大了。
%\end{faq}
%
%
%\begin{faq}{表格如何分页?}
%
%这个问题可见跨页长表格。
%\end{faq}
%
%
%\begin{faq}{表格怎样可以旋转90度?}
%
%希望旋转90度的表格多半是由于过宽而需要进行横排,这里一个方法是使用
%rotating 宏包,使用方法非常简单,用 sidewaytable 替代 table
%即可,但这种表格不能实现跨页长表格(当然又宽又长的表格确实很少见);另一个方法是使用lscape
%宏包提供的 landscape
%环境,进入横排状态,在其中使用相应的环境即可,这种方法可以实现跨页表格,但进入和退出landscape环境时总是会新开一页再进行排版,因此,在其之前的页面可能会留有大量的空白。两种方法各有利弊,可以根据实际需要进行选择。
%\end{faq}
%
%
%\begin{faq}{如何使用图表目录?}
%
%\listoftables
%\listoffigures
%\end{faq}
%
%
%\begin{faq}{图表如何使用双语标题}
%
%使用 bicaption 宏包或 ccaption 宏包。
%\end{faq}
%
%
%\begin{faq}{如何产生表格的竖线,在模板的三线表中产生竖线?}
%
%竖线的产生与否与表格的环境无关,在定义表格列时以 \textbar{}
%分隔列格式即可产生竖线。
%
%\begin{tabular}{l|c|r|}
%  ...
%\end{tabular}
%\end{faq}
%
%
%\begin{faq}{如何在长表格\{longtable\}环境中设置文字自动换行或者固定列宽?}
%\end{faq}
%
%
%\begin{faq}{如何实现表格的奇偶行不同的颜色,长表格也要适用。}
%\end{faq}
%
%
%\begin{faq}{如何使表格单元的左对齐?}
%
%不知道这个问题是啥意思。。。
%\end{faq}
%
%
%\begin{faq}{表格中如何划对角线?}
%
%有宏包 slashbox 或 diagbox 可以制作表格对角线,不过slashbox
%由于没有明确的自由许可信息,已经不为 TeXLive
%所收录了。一个好消息是:diagbox
%有中文版的说明文档,作者的说明总比这里的说明更为准确,直接查阅宏包文档是更好的选择。
%\end{faq}
%
%
%
