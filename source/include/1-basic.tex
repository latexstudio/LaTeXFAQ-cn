\section{背景知识与基本概念}

\begin{faq}{什么是 \TeX{}?}
  \TeX{} 是由著名的计算机科学家 Donald~ E. Knuth(高德纳)发明的排版系统,他在他的书的前言中曾提到%
  \zhquote{\TeX{} 旨在创造美丽的书籍,特别是那些包含很多数学公式的书}。(如果说 \TeX{} 仅仅是为了更
  方便些数学方面的书而生的,那么它就不会像今天这么使用广泛了:事实上, \TeX{} 是一个很好的文字排版
  工具)

  Knuth 是美国加州斯坦福大学计算机编程专业的名誉教授,他在 1978 年开发了第一个版本的 \TeX{} 用来处理
  他的《计算机编程艺术》的修订。这个想法特别受欢迎,所以1982年他推出了 \TeX{} 的第二个版本,也就是
  人们今天所用的 \TeX{} 的基础。

  Knuth 开发了一套\zhquote{文学编程}系统来编写 \TeX{},他还提供了免费的 \TeX{} 资源,以及可以将网络
  资源转化为可以编译或者打印的东西的工具;Knuth 做了什么,对人们来说从来都不是什么神秘的事情。
  \TeX{} 以及它的文档都是高度可移植的。

  对于感兴趣的程序员来说,\TeX{} 有其迷人之处:没有什么能比得上一个人可以构建这样一个程序,至今它的
  持续时间比大多数的程序都好,而且已经被移植到了许多不同的计算机构架和操作系统中——许多现代编程所追求
  的属性。经过处理的\zhquote{可读}的 \TeX{} 程序源代码可以在 TDS structured 版本中找到。

  % 来源 https://texfaq.org/FAQ-whatTeX
\end{faq}

\begin{faq}{\TeX{} 中常见术语的解释}
\textbf{引擎}
  
\LaTeX/\TeX{}解析引擎,其实就是一个编译器,它输入一个\verb|.tex|文件作为输入,根据源文件的内容送入解析引擎和渲染引擎进行处理,并将排版的成果——文档编译输出,\LaTeX/\TeX{}的解析引擎目前有pdflatex、xelatex、lualatex等,它们都可以输出pdf文档文件(部分解析器可以输出dvi文件),用于在多平台进行分发甚至打印出版。

\textbf{格式}

\TeX{} 是存在各种不同的封装格式的,比如原生的 \TeX{} 或者 \LaTeX{},我们所使用的 \LaTeX{} 只是\TeX{} 封装格式的其中一种,是目前流行的封装规范。

\textbf{发行版}

\LaTeX/\TeX{}都包含了成千上万个宏包,甚至有可能我们需要安装新的宏包,除了手动安装外,最好的方式就是利用发行版的宏包管理器,所谓发行版就是把\LaTeX/\TeX{}的相关组件打包,形成一个独立完善的\LaTeX/\TeX{}系统,目前流行的发行版有MiKTeX、proTeXt 以及TeXLive。
\end{faq}


\begin{faq}{}

\end{faq}
