\section{用法惯例}
%
%
%\begin{faq}{TeX编辑器中的魔法注释}
%
%在TeX中有单行注释命令为\%,其后的文本主要是对源代码进行一些说明,它们会被TeX,LaTeX等排版引擎所忽略。但有些注释对专门的TeX相关编辑器来说,可能用特别的意义。在不同的TeX编辑器中,这魔法注释(magic
% comments)可能是不同的。 下面是一些例子: * 指定TeX编译器
%
%\begin{verbatim}
%TeXStudio,TeXworks, Sublime Text
%% !TeX program = xelatex
%
%TeXShop
%%!TEX TS-program = xelatex
%\end{verbatim}
%
%同理,将 xelatex 变为 pdflatex,就可以强制调用 pdfLaTeX 编译器。
%在代码中需要使用ifxetex宏包进行条件判断。 * 指定文档为 utf-8 编码
%
%\begin{verbatim}
%TeXworks,TeXStudio, Sublime Text
%% !TeX encoding = utf8
%\end{verbatim}
%
%Winedt(由于Winedt对编码自动识别能力较弱,使用此注释比较必要,不然要手动设置)
%
%\begin{verbatim}
%% !Mode:: "TeX:UTF-8"
%或者
%% -*- coding: utf-8 -*-
%
%TeXShop
%%!TEX encoding = UTF-8 Unicode
%\end{verbatim}
%
%\begin{itemize}
%  
%  \item
%  指定主文档
%\end{itemize}
%
%\begin{verbatim}
%TeXStudio, Sublime Text
%% !TeX root = filename
%\end{verbatim}
%
%若需要指定上一层次的文件,则应该使用以下命令
%
%\begin{verbatim}
%TeXStudio, Sublime Text
%% !TeX root = ../main.tex
%\end{verbatim}
%
%用两点表示返回上一层次,如果还需要再返回一个层次,则需要
%
%\begin{verbatim}
%TeXStudio, Sublime Text
%% !TeX root = ../../main.tex
%\end{verbatim}
%
%\begin{itemize}
%  
%  \item
%  指定bib处理程序
%\end{itemize}
%
%用biber处理bib文件,可在文件头添加如下代码
%
%\begin{verbatim}
%TeXStudio
%% !TeX TXS-program:bibliography = txs:///biber
%\end{verbatim}
%
%将biber改为bibtex,就可指定bibtex处理bib文件。 * 为TeX编译器指定参数
%
%有时在使用某些宏包时我们需要额外调用一些编译参数,例如 minted
%宏包需要使用 --shell-escape,这时可用如下魔法注释实现该功能
%
%\begin{verbatim}
%TeXStudio
%% !TeX TXS-program:compile = txs://xelatex/[--shell-escape]
%
%sublime text - latextools
%%!TEX options = --shell-escape
%
%texshop
%\end{verbatim}
%
%MathJax是一个JavaScript引擎,用来显示网络上的数学公式。它支持LaTeX、MathML、AsciiMath符号。查阅MathJax支持的LaTeX命令请参考\href{http://http//onemathematicalcat.org/MathJaxDocumentation/TeXSyntax.htm}{http//onemathematicalcat.org/MathJaxDocumentation/TeXSyntax.htm}
%
%\begin{verbatim}
%-shell-escape
%\end{verbatim}
%
%下面是各种编辑器对魔法注释的支持情况
%\textbar{}Encoding\textbar{}Program\textbar{}Root\textbar{}Spellcheck\textbar{}
%:----\textbar{}:----\textbar{}:----\textbar{}:----\textbar{}:----\textbar{}
%TeXShop\textbar{}x\textbar{}x\textbar{}x\textbar{}x\textbar{}
%TeXStudio\textbar{}x\textbar{}x\textbar{}x\textbar{}x\textbar{}
%TextMate\textbar{}?\textbar{}x\textbar{}x\textbar{}?\textbar{}
%TeXworks\textbar{}x\textbar{}x\textbar{}x\textbar{}x\textbar{}
%SublimeText \textbar{}x\textbar{}x\textbar{}x\textbar{}x\textbar{}
%VSCode\textbar{}x\textbar{}x\textbar{}?\textbar{}?\textbar{}
%Atom\textbar{}o\textbar{}x\textbar{}x\textbar{}o\textbar{} Vim
%(vimtex)\textbar{}o\textbar{}x\textbar{}x\textbar{}o\textbar{}
%Texpad\textbar{}o\textbar{}o\textbar{}?\textbar{}?\textbar{} 注意:
%x:支持特性 o:不支持特性
%?:不确定\textbar{}\textbar{}\textbar{}\textbar{}\textbar{}
%
%\begin{itemize}
%  
%  \item
%  其它
%\end{itemize}
%\end{faq}
%
%
%\begin{faq}{LaTeX与数学软件(Mathematica,
%  Maple,Sagemath等)}
%\end{faq}
%
%
%\begin{faq}{LaTeX与公式编辑器}
%\end{faq}
%
%

