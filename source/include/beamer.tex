% Copyright (C) 2018 by latexstudio <http://www.latexstudio.net>
%
% This program is free software: you can redistribute it and/or modify
% it under the terms of the GNU General Public License as published by
% the Free Software Foundation, either version 3 of the License, or
% (at your option) any later version.
%
% This program is distributed in the hope that it will be useful,
% but WITHOUT ANY WARRANTY; without even the implied warranty of
% MERCHANTABILITY or FITNESS FOR A PARTICULAR PURPOSE.  See the
% GNU General Public License for more details.
%
% You should have received a copy of the GNU General Public License
% along with this program.  If not, see <http://www.gnu.org/licenses/>.
%

% !TEX root = ../latex-faq-cn.tex
% !TEX program = xelatex
% !TEX options = --shell-escape

\section{Beamer篇}


\faq{129.隐藏导航栏}{high-NavigationBar}

Beamer
自带的导航符号看起来很不错,但是实际上使用的并不多,为了让文稿的显示面积增加,减少干扰元素,我们可以隐藏下方的导航栏符号,两个方法如下:

\begin{texinlist}
\setbeamertemplate{navigation symbols}{}
\beamertemplatenavigationsymbolsempty % both ok
\end{texinlist}

如果需要去掉下方 title,Author 等信息的话,可以用

\begin{texinlist}
\setbeamertemplate{footline}
\end{texinlist}


\faq{向 Beamer 中添加参考文献}{beamer-add-bib}

我们可以使用下面的命令添加参考文献,最好放在 `appendix' 后面。

\begin{texinlist}
\begin{frame}[allowframebreaks]{References}
\def\newblock{}
\bibliographystyle{plain}
\bibliography{mybib}
\end{frame}
\end{texinlist}


\faq{每节显示目录}{show-toc}

在我们做一个比较长的报告时,我们可能会想在每一节添加一个目录,让听众清楚内容讲到哪了,我们可以在导言区添加如下的命令。

\begin{texinlist}
\setbeamerfont{myTOC}{series=\bfseries,size=\Large}
\AtBeginSection[]{\frame{\frametitle{Outline}%
                  \usebeamerfont{myTOC}\tableofcontents[current]}}
\end{texinlist}

为了得到节的标题信息,我们会在帧与帧之间添加
`\textbackslash{}section{[}short\_title{]}\{long\_title\}', 其中
short\_title 是短标题,用于 ``页眉''
信息(header)显示。如果你不想要显示每帧的页眉信息(header),可以使用下面的命令。

\begin{texinlist}
\setbeamertemplate{headline}{}
\end{texinlist}


\faq{多栏显示}{beamer-mulicol}

有时候我们有图需要并排摆放,一个好方法是使用分栏,尤其是当两个图不同的高度的时候,然后在每一栏插入我们需要的图片。代码如下:

\begin{texinlist}
\begin{columns}[c] % Columns centered vertically.
\column{5.5cm}     % Adjust column width to taste.
\includegraphics ...
\column{5cm}
\includegraphics ...
\end{columns}
\end{texinlist}


\faq{添加 LOGO}{beamer-add-logo}

在右下方添加 logo,直接用系统默认的命令就可以。

\begin{texinlist}
\logo{\includegraphics[width=0.08\textwidth]{logo500}}
\end{texinlist}

如果需要在右上方添加 logo,可以用 TikZ 命令(需要用到 tikz 宏包)在
Frametitle 上添加。

\begin{texinlist}
\addtobeamertemplate{frametitle}{}{%
\begin{tikzpicture}[remember picture,overlay]
\node[anchor=north east,yshift=2pt] at (current page.north east) {\includegraphics[width=0.09\textwidth]{logo500}};
\end{tikzpicture}}
\end{texinlist}


\faq{想在 beamer 中新建一个包含 frame 的环境 question,该怎么做?}{beamer-frame-question}

直接给代码

\begin{texinlist}
\newenvironment{question}
{\begin{frame}[environment=question,fragile]
 \begin{theorem}
}
{\end{theorem}
 \end{frame}
}
\end{texinlist}


% \faq{如何在默认模板的基础上,定制自己的beamer模板}{beamer-default-Customized}


% \faq{如何更改beamer中logo的位置,在使用default的模板和主题下,使用\cs{logo},发现不能更改logo所在位置}{}
